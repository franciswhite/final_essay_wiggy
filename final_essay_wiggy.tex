\documentclass{article}
\title{Wittgenstein and the Savages\\
\large Some Thoughts on the Ethnographical Method in his Later Work}
\date{}
\author{Silvan Hungerbuehler}
\usepackage{enumitem}
\usepackage{verbatim}
\usepackage{hyperref}
\usepackage{comment}
\usepackage[utf8]{inputenc}
\usepackage[english]{babel}
\usepackage{amsmath}


%\usepackage[backend=biber]{biblatex}
%\addbibresource{references_wiggy.bib}
%\usepackage{csquotes}
\begin{document}
\maketitle
\section{Introduction}
Throughout much of his later writing Wittgenstein frequently assumes what might be called an \textit{ethnological stance}. In these moments Wittgenstein creates the illusion of the adventurous explorer visiting an entirely foreign people. In this hypothesiszed position of a complete stranger
%5Here I could use a good amount of Quine/Davidson/Glock description 
he then investigates some completely unknown language, as in PI 206: “[…] Denke, du kämst als Forscher in ein unbekanntes Land mit einer dir gänzlich fremden Sprache.” or in PI 207: “Denken wir uns, die Leute in jenem Land verrichteten gewöhnliche menschliche Tätigkeiten und bedienen sich dabei, wie es scheint, einer artikulierten Sprache.”. At various times he uses the picture of the wild tribe, as in OC 264: “Ich könnte mir den Fall denken, dass Moore von einem wilden Volksstamm gefangen genommen wird[…].” Or this in PI 243: ''Man k{\"o}nntes ich auch Menschen denken, die nur monologisch spr{\"a}chen. [...] Einem  in Forscher, der sie beobachtet und ihre Reden belauscht, k{\"o}nnte es gelingen, ihre Sprache in die unsre {\"u}bersetzen. (Er w{\"a}re dadurch in den Stand gesetzt, Handlungen dieser Leute richtig vorherzusagen, denn er h{\"o}rt sie auch Vors{\"a}tze und Entschl{\"u}sse fassen.)"

At the same time, Wittgenstein wrote an extensive critique of James George Frazer's monumental comparative study of religion and mythology. Frazer's account is definitely an anthroplogical one and some Wittgenstein's remarks thus of a more practical nature. The purpose of the essay at hand is to asses and compare Wittgenstein's views on the actual practice of anthropology - as manifested in his \textit{Remarks on Frazer's Golden Bough} - with the his own use of speculative ethnology. Wittgenstein has made a couple of interesting points delimiting what ethnological work can and can’t do when investigating ritual, mythology and belief. My question then is: How does Wittgenstein’s use of these ethnological thought experiments relate to his remarks on Frazer’s actual ethnographic work?

Since Wittgenstein is certainly no professional anthropologist, a first pertinent question must concern the purpose of his hypothesisez ethnological field trips. In \hyperlink{sec2}{Section 2} I will try to give some rough account of the use Wittgenstein makes of this philosophical device. I will also cursorely come to speak of his reception in the field of anthroplogy. \\
\hyperlink{sec3}{Section 3} will be concerned with tracing Wittgenstein's \textit{Remarks on Frazer's Golden Bough} and their relation to the findings of the previous section. Can Wittgenstein’s views on the proper place of anthropological description as well as his own (albeit merely supposed) ethnographic investigations be accommodated within his wider philosophical views? Finally, \hyperlink{sec4}{Section 4} concludes and \hyperlink{sec4}{Section 5} contains the bibliography.\\
\textbf{So far $\approx$ 370 words} 
\section{Philosophy as Anthropogy}
\hypertarget{sec2}{Hacker, Dromm:}

\paragraph{Intro} Wittgenstein's later philosophy offers a powerful conception of human language use. Scholarly opinion on the correct interpretation of Wittgenstein's writings varies widely. I will not be able to develop a thorough and convincing account myself at the present moment. Therefore I have decided to sketch two interpretations which disagree on what seems to be crucial issue when trying to make sense of Wittgenstein's views on the matters of language use, its acquisition, linguistic meaning etc. The rift or point of disagreement I will focus on, is the role the ascription of a specific kind of \textit{naturalism} to Wittgenseint plays in the two exegetical efforts. First, I will present aspects of Peter Hacker's interpretation, and then, secondly, come to speak of Keith Dromm's take. Interestingly both interpreters offer an interesting - albeit vastly different - basis from which to assess Wittgenstein's comments on the work of Frazer. As that is what I am ultimately interested in, my rendering of both Dromm and Hacker will be quite selective.
\paragraph{Hacker} Peter Hacker's interpretation, for one, sees language use as a variety of rule-governed activities in human communities. Within and for these \textit{language-games}, interlocutors learn the correct application of concepts and response to their uttering of words, how to give explanations appropriate to the context etc. In short, the learning of a language consists in mastering the technique its usage. \\
Just as human beings use tools in the context of their everyday activities, language users perform acts with words in the context of their communal lives.

Hacker points to sections as the following to sustain his views:
''Das Wort "Sprachspiel" soll hervorheben, dass das \textit{Sprechen} der Sprache ein Teil ist einer T{\"a}tigkeit, oder einer Lebensform.''\\ %\cite{wittgenstein2009philosophische} \\
So human language is conceived to be centered in practice. This practice takes place in the \textit{flow of life}, in the human community of family at first and broader society later. The concepts employed across different places, times, communities or societal subgroups arise from a shared way of living. From elementary human needs to highly contingent circumstances of one particular form of co-existence and evertything between these extremes, all these factors go into producing the particular concepts used by some group of humans. Against Hacker he holds that Wittgenstein's point is not that humans learn and use language because they share it with other speakers

The reason Wittgenstein frequently recurs to an \textit{anthropological} or \textit{ethnological} stance when describing and reflecting on philosophical issues is intimitaly tied to this conception of human language. For the fact that words and phrases have meaning at all, is tied to the fact that there are rules governing the correct us of words. If there were no standards to discern correct from incorrect application of a word, it would be meaningless. %(Cf.\cite{glock1996safari})
Thus acquiring and using a language is in a very important sense a normative practice. The community of language users constantly shapes what the language-games in use, expands old ones and abolishes moves deemed once legal.

If one follows this rough interpretation of Wittgenstein which stresses the rootedness of language and meaning in practical, natural context within which it is being put to use, then it would come quite naturally to study language as a quasi-anthropologist - as somebody who studies humans.

So concepts are human creations, governed by rules stipulated, controled and inculcated by humans. As P.M. Hacker points out, however, it seems that Wittgenstein is not primarily interested in analysing the actual history of the concepts familiar to him. (a possible exception might be his work in \textit{On Certainty}). So although he argues for the tantamount importance of viewing the grammar of the language we happen to do philosophy with as a historical product, he does not seem to see himself as a historian of that grammar. Hacker uses the slogan: ''Historicism without history."\\
Wittgenstein reflects on the relation between concept-formation and historical considerations in the Philosophy of Psychology:
%\cite{hacker2010wittgenstein}

Hacker: That human beings use language, engage in language-games, perform acts of speech in the context of their activities - these are anthropological facts about the nature history of man. What warrants using the epithets ''ethnological approach'' or ''anthropological approach'' in describing Wittgenstein's later philosophy is the perspective from which he view conceptual matters. 
\paragraph{Dromm}
\textbf{We can resist the temptation to read Wittgenstein's as making empirical or naturalistic claims of some sort in his later writings by, first, recognizing the scope of his interest. He is interested in the grammar of our language-games for the purpose of deciding on answers to questions about scepticism, meaning, understanding, rationality etc. The observations he makes about grammar hold regardless of the truth of the empirical theories he describes. Seoncd, to understand the role played by these empirical theories requires our seing how simply being able to imagine them tells us something important about the grammar of the language-game for which they are imagined.}
Of course, there are various points where Keith Dromm disagress with Hacker. The contention I would like to focus on, however, concerns the the interpretation of the employment of terms like ''natural'' and ''primitive'' in Wittgenstein's writings. According to Dromm Wittgenstein should not be understood to issue empirical claims about biology, the nature of human societies etc. He is not referring to facts about the human physical or societal condition when he talks of ''natural history'' (PI 25,415). Dromm claims that although Wittgenstein does offer a theory of language, this must be understood as him demonstrating one \textit{possible} theory. Wittgenstein does not put it forward hoping that it will be accepted as true, as the Dromm takes his opponents to claim, but merely shows the \textit{possibility} of this specific account. Wittgenstein doesn't affirm one theory, he merely invites his readership to imagine one possible account. And by imagining one possible theory of language, essential features of the phenomenon in question are made salient. By imagining different possible forms of life where different language-games are being played.\\
Cause and Effect: The basic form of the game can't include doubt. What we are doing here above all is to imagine a basic form: a possibility, indeed a very important possibility.(We very often confuse what is an important possibility with historical reality.)\\

Dromm moreover claims that these hypothetical language-games, especially regarding terms like ''natural history" or ''form of life", are no empirical claims related to the study of human behavior. Dromm takes them only to describe the fact that the language-games we know could also be thought of differently.

\textbf{That this is one possibility, that we can imagine that our language developed in this way, is supposed to compel us to recognize those important grammatical features of first-person sensation talk that I reviewed before. ... The possibility Wittgenstein describes is not meant to contribute to an understanding of linguistic development or acquisition - what interest would this serve? - but to highlight the similar grammars of the basic form and our more sophisticated language-game. p. 683}

\textbf{And he is not saying that these moves (i.e primitive reactions) are historically antecedent to the language-game; rather, they are logically antecedent. p. 685}

\textbf{Concerning justification: The key to understanding this, is not to take it to be saying something about human behavior - as Hacker might want it - but instead as saying something about the grammar of a language-game. p.686\\
But Wittgenstein never told how or where justifications end; only that they will end somewhere. p. 687}
\paragraph{Synthesis, my own 50 cents, what to anthro?}
Personally, I do not follow Dromm in his conclusion. It seems to me that Wittgenstein goes to great length to give a positive account of how language might depend on facts about the way humans are and behave and it appears contrived to me to claim that this effort was only to show the possibility of giving some such account. However, the question of who is right on this count is irrelevant to the purpose at hand, for I wholeheartedly agree with Dromm that the device of inviting his readership on an imagined field trip to the primitives does serve a crucial function. 

\section{Anthropological Remarks on Frazer}
\hypertarget{sec3}{Frazer}
\paragraph{Intro - Frazer's condescending stance, Wiggy's use of ''savage", Imagine differently}
The following section aims to reproduce certain key comments Wittgenstein made in his \textit{Remarks on Frazer's Golden Bough}. Whether Wittgenstein's reading of Frazer's word is fair or accurate bears no importance on the present essay. I would merely like to extract a reading of passages related to Wittgenstein's own use of anthropological thought experiments to illuminate their usefullness in light of the larger context of his philospohy of interpretation and language. 

According to the standard we apply to the \textit{Golden Bough}, that is, Wittgenstein's reading of it, it is clear that Frazer's goal is to give an account of ''\textit{mankind}" as its subject. It is thus a work of anthropology. Wittgenstein advances different points of criticism towards it, some of them explicit, others in the form of thinly veiled sarcasm. A nice example of the latter can be found inhis statement that Frazer is \\
''[...] much more savage than most of his savages[...]''.\\ %\cite{rfg}p. 68
 Indeed, Wittgenstein seems to view a great deal of condescension in how Frazer describes and reasons about the people he studies. Consider the following passage, for example, when Wittgenstein first claims that Frazer's conceptual assumptions behind his attempt to provide \textit{explanations} of foreign practices are misguided:\\
 \begin{quote}
 ''The very idea of wanting to explain a practice - for example, the killing of the priest-king - seems wrong to me. All that Frazer does is to make them plausible to people who think as he does. It is very remarkable that in the final analysis all these practices are presented as, so to speak, pieces of stupidity.''\\ %\cite{rfg}p.61
 \end{quote}
 There are various passages like the one just given in which Wittgenstein calls Frazer out for basing his claims on the assumption that the people under study are somehow culturally - or even cognitively - deficient.
 
In the light of the purpose of this essay perhaps the most crucial aspect of Wittgenstein's attack on Frazer's stance towards the objects he studies is ''how impossible it was for him to conceive of a life different from that of England of his time.''. A series of remarks to this effect play an important role in the relation between anthropological thought experiments and Wittgensteins's critism of Frazer's ethnological accounts, I will come to speak of them in more depth in a moment.
 Despite his criticism it must be conceded, however, that Wittgenstein also adopts the term ''savages'' - albeit reluctantly. Some commentators have suggested that this is part of a an attempt to re-coin the disparaging term or an ironic usage of it. %\cite{bookonwiggyandfrazer}
 
 
\paragraph{Glance at criticism of Frazer: poor explanation as explanation, misguided explanation bc of wrong dichotomy of belief and act}
I want to take a brief look at Wittgenstein's main points of contention with Frazer's account of religious and magical views and practices across the world. There are principally two types of criticism concerning Frazer's analyisis of ritual practices. 

The first type of critique accepts Frazer's claim that there are two seperate constitutive elements of a ritual. Namely, a practical part and a theoretical or belief-related part that serves an explanatory purpose. There is a supposed dichotom between the acts - motions, gestures, chants etc. - that are carried when the ritual is performed on the one hand, and the beliefs - a historical narrative or some underlying explanation, in a way like a spiritual mechanics. Frazer holds that the acts performed throughout the ritual follow from the beliefs and explanations. In order to provide an explanation of the rituals in question, Frazer thus ascribes beliefs to the people whom he observes performing some ritual. On his account ritual and magic practices are some kind of pseude-science, a primitive prototype that has yet to evolve to the more refined scientifc method known in England at the time. 

Supposing this dichotomy, that practice and theory can be intelligibly seperated, then, Wittgenstein replies that this is simply a bad explanation of the phenomena in question. For it supposes that the people performing the rituals hold vastly erroneous beliefs across the board.
\begin{quote}
Frazer's account of the magical and religous views of mankind is unsatisfactory: it makes the views look like \textit{errors}.
\end{quote} 
This is surely related to Frazer's condescension: In order for his explanation to work he needs to assume that the people whose rituals he analyses are too ignorant to realize the blatant stupidity of their beliefs. Wittgenstein provides a fine example of how little such an explanatory schemes actually explains when noting the bizarrity of ascribing the belief that a ritual will causally bring about rain to people performing said ritual right at the start of the rain-season.
%The point Wittgenstein makes here might be be seen in the light of Donald Davidson's \textit{principle of charity}. According to it, it is impossible to make sense of what some completely unknown group is doing unless one ascribes mostly true and rational beliefs to them.

The second, and probably more pertinent, point of contention concerns not the internal consistency of Frazer's attempt at providing an analysis of rituals in two seperate parts. Wittgenstein rejects the seperation of rituals into actions and views altogether, instead, he holds that views and practices simply co-occur. He sees no sense in trying to extract some explanation, possibly by ascribing some view according to which the ritual-action performed would fullfil a causal purpose in the belief-system in place.

\textbf{Here I need to invest more and work out the finer details. Especially wrt where I want to go: thought experiments (even wildest imaginations is still intellible)}
What is meaningful about ritual is something that survives change - god did not help me, therefore I will stop praying. Our ability to understand the ritual practices is not dependant on the fact that we are ourselves a kind of pseudo-scientist or that we don't understand the proper causal connections that are in place. Wittgenstein speaks of creating a sense of relief. Walking stick exampole. 

Furthermore, the analyisis of rituals, the understanding of rituals is in essence understanding of what we aer, irrespective of cultural differences. A particular culture or civilization is just some specific and contingent expression of fundamental, absolute human values.

\paragraph{Key passage: Frazer can't imagine that it could be different, inventing primitive practices}
This leads to the main thing I would like to extract from Wittgenstein's remarks on Frazier. Right after giving a series of examples where reading one or another ritual  in a scientific manner seems ludicrous to him, Wittgenstein makes the following exclamation:
\begin{quote}
[H]ow impossible it was for [Frazer] to conceive of a life different from that of England of his time! [...] Frazer cannot imagine a priest who is not basically a present-day English parson with the same stupidity and dullness.
\end{quote}
I want to argue that several issues are at stake here. For one, Wittgenstein's issue with Frazer's inability to shed his personal point of view to a sufficient degree goes back to the criticism of Frazer's ethnological work. That Frazer takes what is known to him and ineptly transfers it to construct poor explanation, that is, bad even on his own terms, of the phenomena he writes about. A further issue goes to the heart of the matter of what this essay is attempting to convey. Namely, that Wittgenstein's exasperation with Frazer's lack of imagination is related to a tool or method Wittgenstein likes to use in his own philosophical work. And\\
 ''to conceive of a life different'' is the central element by which the tool functions. So to study the attack on Frazer in this respect goes quite a long way in making sense of Wittgenstein's use of fabricated anthropological scenarios throughout the PI or the OC. 
 
There are three crucial claims Wittgenstein makes one paragraph below the passage quoted above. First, that anything we could come up with when imagining possible practices, they might just exist.
\begin{quote}
One sees how misleadig Frazer's explanations are [...] by noting that one could very easily invent primitive practices oneself, and it owuld be pure luck if they were not actually found somewhere.
\end{quote}
So, Wittgenstein notes, there is no principled difference between his thought experiments about how a society could be constituted and actual anthropological data. Imagination serves just as well.
The second claim holds that the principle - the underlying traits of human beings involved in rituals - follow some pattern or \textit{general principle}. There is a common thread through all societies and their rituatlistic practices. This quote follows the last one immediately.
\begin{quote}
That is, the principle according to which these practices are arranged is a much more general one than in Frazer's explanation and it is present in our own minds, so that we ourselves could think up all the possibilities.
\end{quote}
The third claim is related to the first and can be found in the latter part of the quote just given. All the imagined scenarios we may construct somehow instantiate or follow the same \textit{general principles} embedded in all human minds. 

When putting these claims together we obtain a valid tool to investigate ''the principle according to which these practices are arranged'' of all the ritualistic practices of humankind. Without even having to check whether some conjured up example are or were actually existing at some point and place we can investigate the overarching, structural commonalities of all these rituals. The reason this is a valid and generally applicable method, according to Wittgenstein, is because we - human beings - are all in possession of the same kind of mind and, in virtue of this equally powerful mind, in the position to imagine all possible practices.

\textbf{TLP: isomorphism between structure of the world and the structure of thought.
Wittgenstein's exploration of this ramification of his thought is very brief at this point in time. We might conjecture that some remnant from the Tractatus' idea that thought and world are structurally isomorphic lies behind Wittgenstein's confidence in the universal extension of imaginative powers.}
\paragraph{human generalities, kinship with savages}
In addition to the very strong claims given in the last paragraph Wittgenstein tries at multiple times throughout the remarks on Frazer to argue for the fundamental likeness between ''us'' and ''the savages''. In some passage he speaks of ''kinship'' between people and he seems to mock Frazer at various points for trying to construct a difference where there really is none to be found. 

Shortly after the threefold claim concerning the universal access to the realm of possible rituals he offers the following commonality between all humans. I is especially pertinent to the argument because it is more concrete as to what commonalities he has in mind and how these could lead to different people inventing practices that are similar in fundamental structure.
\begin{quote}
There are dangers connected wiht eating and drinking, not only for savages, but also for us; nothing is more natural than the desire to protect oneself from these; and now we could devise such a preventative measure ourselves. - But according to what principle are we to invent them? Obviously, according to the one by which all dangers are reduced to the form of a few very simple ones which are immediately evident to man.
\end{quote}
\textbf{elaborate!\\ 
fit this in somewhere: That is, one could begin a book on anthropology by saying: When one examines the life and behavior of mankind throughout the world, on sees that, except for what might be called animal activities, such an ingestion, etc., etc., etc., men also perform actions which bear a characteristic peculiar to themselves, and these could be called ritualistic actions.}


\section{Conclusion}
\hypertarget{sec4}{arrg}
\section{References}
\hypertarget{sec5}{grr}
%\printbibliography




\end{document}