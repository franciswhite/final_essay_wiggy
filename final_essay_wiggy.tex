\documentclass{article}
\title{Wittgenstein and the Savages\\
\large Some Thoughts on the Ethnographical Method in his Later Work}
\date{}
\author{Silvan Hungerbuehler}
\usepackage{enumitem}
\usepackage{verbatim}
\usepackage{hyperref}
\usepackage{comment}
\usepackage[utf8]{inputenc}
\usepackage[english]{babel}
\usepackage{amsmath}


%\usepackage[backend=biber]{biblatex}
%\addbibresource{references_wiggy.bib}
%\usepackage{csquotes}
\begin{document}
\maketitle
\section{Introduction}
%primitive: PI 554
%tribe: PI 2, 200, 282, 385, 419
%savage: already quoted
A curious site presents itself to the reader of the later Wittgenstein. Is Wittgenstein already going on expedition again? Indeed, the professor has stuffed the mosquito net, sturdy safari helmet and machete into a brown briefcase and is heading for his leather armchair. From there, he proclaims, he shall investigate the most exotic peoples known and unknown to man. How does Wittgenstein deem this possible, one might ask. The present essay is an attempt to offer some take on why Wittgenstein's manifold anthropological ventures never extended - in fact, never had to extend - beyond the convenience of his office. 

Throughout his later writing Wittgenstein frequently assumed what might be called an \textit{ethnological stance}. In these moments Wittgenstein mimicks the anthropologist visiting an entirely foreign people to study life and behavior of the humans there to be found. In the hypothesiszed position of a complete stranger
%5Here I could use a good amount of Quine/Davidson/Glock description 
he investigates some phenomenon he himself conjured up. Many examples include investigations into aspects of a completely unknown language, as in the \textit{Philosophical Investigations} (PI) §206:
\begin{quote}
Suppose you came as an explorer into an unknown country with a
language quite strange to you. In what circumstances would you
say that the people there gave orders, understood them, obeyed them,
rebelled against them, and so on?
\end{quote}
or this in PI §243:
\begin{quote}
We could even imagine human beings who spoke only in monologue; who accompanied their activities by talking to themselves. —An explorer who watched them and listened to their talk might succeed in translating their language into ours.
\end{quote}
or in PI §207: 
\begin{quote}
Let us imagine that the people in that country carried on the usual human activities and in the course of them employed, apparently, an articulate language.\end{quote}
 At various times he uses the picture of the wild tribe to somehow imagine customs that are a far cry from what he and his contemporary peers were used to, as in OC §264, when he even puts Moore in some hypothesized confrontation with the ''savages'':
 \begin{quote}
Ich könnte mir den Fall denken, dass Moore von einem wilden Volksstamm gefangen genommen wird[…].
\end{quote}

At the same time, Wittgenstein has written an extensive critique of James George Frazer's monumental comparative study of religion and mythology he found throughout the world. Frazer's account is definitely one of practical nature, while there is no indication that Wittgenstein ever engaged in actual field work and stuck to hypothesizing. This then makes Wittgenstein's comments on Frazer's work a critique of practical anthropology and the question arises how these are related to his own purely ideal approach. The central thesis of this essay is that for Wittgenstein's philosophical intentions there is no sensible distinction between classical, laboriously collected, anthropological data and insights gathered from thought experiments. To make this plausible I want to assess and compare Wittgenstein's views on the instance of Frazer's actual practice of anthropology - as manifested in his \textit{Remarks on Frazer's Golden Bough} - with his own use of speculative ethnology. To slightly rephrase my question: How does Wittgenstein’s use of these ethnological thought experiments relate to his remarks on Frazer’s actual ethnographic work?

Quite evidently Wittgenstein's primary business does not at all coincide with what anthropologist usually do. He does not engage in any of the tasks that is normally associated with that field (describing people, customs, social structure, culture or attempting to make comprehensible or ''translating'' between worldviews etc.). A first relevant question, therefore, must concern the purpose of his hypothesised ethnological field trips. \textbf{In \hyperlink{sec2}{Section 2} I will try to give some rough account of the use Wittgenstein makes of this philosophical device.} %What am I exactly doing here.
 I will also cursorely come to speak of his reception in the field of anthroplogy. \\
\hyperlink{sec3}{Section 3} will be concerned with tracing Wittgenstein's \textit{Remarks on Frazer's Golden Bough} and their relation to the findings of the previous section. Can Wittgenstein’s views on the proper place of anthropological description as well as his own supposed ethnographic investigations be accommodated within his wider philosophical views? Finally, \hyperlink{sec4}{Section 4} concludes and \hyperlink{sec4}{Section 5} contains the bibliography.

\section{Philosophy as Anthropogy} %hypertarget{}
\paragraph{Intro} Wittgenstein's later philosophy offers a powerful conception of human language use. Scholarly opinion on the correct interpretation of Wittgenstein's writings varies widely. At the present moment, I will obviously not be able to develop a thorough or convincing account myself. In the light of the discussion of Wittgenstein's remarks on Frazer, however, a rough presentation of some exegetical views is pertinent. Therefore, I have decided to sketch two interpretations which overall disagree on many counts how Wittgenstein ought to be read. Most important to the present essay, however, is their rift on the role of Wittgenstein's views on the matters of language use, acquisition, linguistic meaning etc. - short and overly simple: his philosophy of language (and whatever is tied to it) - play in the context of the whole of his philosophy.

First, I will present aspects of Peter Hacker's interpretation and, secondly, come to speak of Keith Dromm's take. Both interpreters offer an interesting - albeit vastly different - basis from which to assess Wittgenstein's comments on the work of Frazer. As that is what I am ultimately interested in, my rendering of both Dromm and Hacker will be quite selective.

\paragraph{Hacker} Peter Hacker's interpretation, for one, sees language use as a variety of rule-governed activities in human communities. Within and for these \textit{language-games}, interlocutors learn the correct application of concepts and response to their uttering of words, how to give explanations appropriate to the context etc. In short, the learning of a language consists in mastering the technique of its usage.
Just as human beings use tools in the context of their everyday activities, language users perform acts with words in the context of their communal lives. 

Hacker points to sections as the following from PI 23 to sustain his reading:
\begin{quote}
''Here the term \textit{''language-game''} is meant to bring into prominence
the fact that the \textit{speaking} of language is part of an activity, or of a form
of life.''
\end{quote} %\cite{wittgenstein2009philosophische} \\
Thus human language is conceived to be centered in practice. This practice takes place in the \textit{flow of life}, in the human community of family at first and broader society later. The concepts employed across different places, times, communities or societal subgroups arise from a shared way of living. From elementary human needs to highly contingent circumstances of one particular form of co-existence and everything between these extremes, all these factors go into producing the particular concepts used by some group of humans.
On this view of language, the fact that words and phrases have meaning at all, is tied to the existence of rules governing the correct us of words. If there were no standards to discern correct from incorrect application of a word, it would be meaningless. 
%(Cf.\cite{glock1996safari})
Thus acquiring and using a language is in a very important sense a normative practice. The community of language users constantly shapes what the language-games in use, expands old ones and abolishes moves deemed once legal.
So concepts are human creations, governed by rules stipulated, controlled and inculcated by humans. \begin{quote}
It helps one to view the normative grammatical structures that inform a language as a net, to see it as a human artifact that could have been woven differently, to realize its normative role in the natural history of a human language-using community [...].
\end{quote}
The task of philosophy, as Hacker views it, should then be to disentangle conceptual issues – knots in this net of norms - from which philosophical problems arise.

 Calling Wittgenstein's later approach to philosophy ''ethnological'' or ''anthropological''is thus warranted by the nature of the problems it deals with. Namely, the contingent and community based formation of a web of norms regulating language practice. However, philosophy should not be construed as simply a subbranch of anthropology. The ‘’ethnological’’ viewpoint merely serves as a tool to distance oneself from the phenomena under philosophical investigation. It aides in conducting philosophy in more ‘’objective’’ fashion.
To sum up the main point of this, Hacker takes Wittgenstein to put forward some linguistic theory that explains the emergence of meaning, language acquisition and so forth while only using the ‘’ethnological’’ method as a tool to construct this theory.

\paragraph{Dromm}
According to Keith Dromm's interpretation, Wittgenstein in no way advances anything resembling ''empirical or naturalistic claims'' in his later work. What Hacker thinks is an account - albeit a very general one - of the how language is acquired by children etc., is simply not that. Dromm believes that although Wittgenstein is trying to respond to questions about meaning, rationality, understanding or scepticism he does so irrespective of the empirical theories one might find in his later writings.

Wittgenstein should not be understood to issue empirical claims about biology, the nature of human societies and so forth. He is not referring to facts about the human physical or societal condition when he talks of ''natural history'' (PI 25,415). Dromm claims that although Wittgenstein seems to offer a theory of aspects of language, this must be understood as him demonstrating one \textit{possible} theory. Wittgenstein does not put it forward hoping that it will be accepted as true, as the Dromm takes his opponents to claim, but merely shows the \textit{possibility} of this specific account. Wittgenstein doesn't affirm one theory, he merely invites his readership to imagine one possible account. And by imagining one possible theory of language, essential features of the phenomenon in question are made salient. Dromm points to a quote in \textit{On Cause and Effect} to this effect:
\begin{quote}
The basic form of the game can't include doubt. What we are doing here above all is to imagine a basic form: a possibility, indeed a very important possibility.(We very often confuse what is an important possibility with historical reality.)
\end{quote}
By imagining different possible forms of life where different language-games are being played. These hypothetical language-games, especially regarding terms like ''natural history" or ''form of life", are thus not empirical claims related to the study of human behavior. They should only describe the fact that the language-games we know could also be thought of differently.
\begin{quote}
The possibility Wittgenstein describes is not meant to contribute to an understanding of linguistic development or acquisition - what interest would this serve? - but to highlight the similar grammars of the basic form and our more sophisticated language-game. p.681
\end{quote}

So Wittgenstein's anthropological ventures are, in Dromms view, not a way to develop some empirical theory of any kind, but merely serve the purpose to investigate the grammer of language-games.
%\textbf{Concerning justification: The key to understanding this, is not to take it to be saying something about human behavior - as Hacker might want it - but instead as saying something about the grammar of a language-game. p.686\\
%\paragraph{Synthesis, my own 50 cents, what to anthro?}
%Personally, I do not follow Dromm in his conclusion. It seems to me that Wittgenstein goes to great length to give a positive account of how language might depend on facts about the way humans are and behave and it appears contrived to me to claim that this effort was only to show the possibility of giving some such account. However, the question of who is right on this count is irrelevant to the purpose at hand, for I wholeheartedly agree with Dromm that the device of inviting his readership on an imagined field trip to the primitives does serve a crucial function. 

\section{Anthropological Remarks on Frazer}
%\hypertarget{sec3}{Frazer}
\paragraph{Intro - Proper Reading, Frazer's condescending stance, Wiggy's use of ''savage''}
The following section uses a small amount of passages from Wittgenstein's \textit{Remarks on Frazer's Golden Bough}to shed some light on Wittgenstein's use of anthropological thought experiments. Moreover, the usage Wittgenstein makes of this device is supposed to shed some light of the larger context of his philosophy. 

It must be noted that whether Wittgenstein's reading of Frazer's work is fair or accurate bears no importance on the present essay. As the interest is neither Frazer's work itself, nor really Wittgenstein's opinion of it as an anthropological study, but merely the insight we can obtain from it into Wittgenstein's own philosophy. 

%According to the standard we apply to the \textit{Golden Bough}, that is, Wittgenstein's reading of it, it is clear that Frazer's goal is to give an account of ''\textit{mankind}" as its subject. It is thus a work of anthropology. 
First it must be noted that Wittgenstein attacks Frazer on several counts, not all of which are directly pertinent now. The modes in which he advances his multitude of critique also vary; some points of critique he voices explicitly, others take the form of thinly veiled sarcasm. A nice example of the latter can be found in his statement that Frazer is
\begin{quote}
''[...] much more savage than most of his savages[...]''.
\end{quote} %\cite{rfg}p. 68
Indeed, Wittgenstein seems to take issue with the condescension he views in how Frazer describes and reasons about the people he studies. Consider the following passage, for example, where Wittgenstein harshly questions Frazer's conceptual assumptions behind his attempt to provide \textit{explanations} practices foreign to him:
 \begin{quote}
 ''The very idea of wanting to explain a practice - for example, the killing of the priest-king - seems wrong to me. All that Frazer does is to make them plausible to people who think as he does. It is very remarkable that in the final analysis all these practices are presented as, so to speak, pieces of stupidity.''%\cite{rfg}p.61
 \end{quote}
 There are various passages like the one just given in which Wittgenstein criticizes Frazer for basing his \textit{explanation} on the assumption that the people under study are somehow - culturally or cognitively - deficient. For fairness' sake it must be admitted, however, that it is not entirely clear Wittgenstein himself was free of such prejudice. At times it appears that he operates with a normative distinction between civilized and (still) culturally primitive people, as in PI 194:
\begin{quote}
When we do philosophy we are like savages, primitive people, who hear the expressions of civilized men, put a false interpretation on them, and then draw the queerest conclusions from it.
\end{quote}
 Also, Wittgenstein adopts the term ''savages'' throughout his remarks on Frazer - albeit he seems to do so reluctantly. Some commentators have suggested that this is part of a an attempt to re-coin the disparaging term or an ironic usage of it. %\cite{bookonwiggyandfrazer}
%If lacking space, just leave this quote out. And leave the horrible italian man out.

As far as this essay is concerned perhaps the most crucial aspect of Wittgenstein's attack on Frazer's stance towards the people that are the objects of his studies is ''how impossible it was for him to conceive of a life different from that of England of his time.''. A series of remarks to this effect play the central role in the relation between anthropological thought experiments and Wittgensteins's critism of Frazer's ethnological accounts. I will come to speak of them in more depth in a moment, but before I would like to give a slightly more elaborate picture of Wittgenstein's views on the anthropological stance in \textit{The Golden Bough}.
 
\paragraph{Glance at criticism of Frazer: poor explanation as explanation, misguided explanation bc of wrong dichotomy of belief and act}
I want to take a brief look at Wittgenstein's main points of contention with Frazer's account of religious and magical views and practices across the world. There are principally two types of criticism concerning Frazer's analysis of ritual practices. 

The first type of critique accepts the claim attributed to Frazer that there are two seperate constitutive elements of a ritual. Namely, a practical part and a theoretical or belief-related part that serves an explanatory purpose. Wittgenstein reads Frazer as to postulate a dichotomy between the acts - motions, gestures, chants etc. - that are (observably) carried out when the ritual is performed on the one hand, and the beliefs - historical narratives or underlying theoretical explanation, in a way like a spiritual mechanics - on the other. Furthermore, the acts performed throughout the ritual follow from the beliefs or views. In order to provide an explanation of the rituals in question, Frazer is thus portrayed as ascribing beliefs to the people whom he observes performing some ritual. This then yields an account of ritualistic and magic practices as some kind of pseude-science, a primitive prototype that has yet to evolve to the more refined scientifc method known in England at the time. 

Supposing this dichotomy, that practice and theory can be intelligibly seperated, then, Wittgenstein replies that this is simply a bad explanation of the phenomena in question. For it supposes that the people performing the rituals hold vastly erroneous beliefs across the board. \footnote{I am reminded of Davidson's thought that, in order to interpret the acts of some unknown group, we need to view their beliefs as largely true and rational.}
\begin{quote}
Frazer's account of the magical and religous views of mankind is unsatisfactory: it makes the views look like \textit{errors}.
\end{quote} 
This ascription of of false beliefs fits well with Frazer's condescension: For his explanation to work he needs to assume that the people whose rituals he analyses are not able to realize the blatant stupidity of their beliefs. Wittgenstein provides a fine example of how little such an explanatory schemes actually explains. He draws attention to the absurdity of ascribing the belief that a rain-ritual will causally bring about rain to people who are performing it right at the start of the rainy season.

The second type of contention concerns not the internal consistency of Frazer's dissolution of rituals into two seperate parts. Wittgenstein rejects the seperation of rituals into actions and views altogether.
In fact, the very attempt of finding the \textit{explanation} Frazer is looking for strikes him as dubious.
Instead, he suggests is to see ritualistic acts not as part of some pseudo-scientific worldview, but as acts somehow directed towards the sensation of relief. Our ability to understand the ritual practices is therefore not dependant on the fact that we are ourselves a kind of pseudo-scientist or that we don't understand the proper causal connections that are in place. Rather the most fruitful stance is to undestand rituals as essentially expressing \textit{what we are} as humans, irrespective of particular cultural differences. A particular culture or group of people with its specific ritual practices is just one contingent expression of fundamental, absolute human values. Wittgenstein's conviction concerning the existence of absolute value, as expressed in the \textit{Lectures on Ethics}, becomes apparent here.

\paragraph{Key passage: Frazer can't imagine that it could be different, inventing primitive practices}
This leads right up to the main element I would like to extract from Wittgenstein's remarks on Frazier. Following a series of examples where Frazer's scientific reading of rituals irks him, Wittgenstein exclaims:
\begin{quote}
[H]ow impossible it was for [Frazer] to conceive of a life different from that of England of his time! [...] Frazer cannot imagine a priest who is not basically a present-day English parson with the same stupidity and dullness.
\end{quote}
Several issues are at stake here: For one, Wittgenstein's issue with Frazer's inability to shed his personal point of view to a sufficient degree goes back to the criticism of Frazer's ethnological work. Frazer takes what is known to him and ineptly transfers it to construct poor theory of the phenomena he writes about, that is, defective explanations even on his own terms. A further issue goes to the heart of the matter of what this essay is attempting to convey. Namely, that Wittgenstein's exasperation with Frazer's lack of imagination is related to a tool or method Wittgenstein likes to use in his own philosophical work. And ''to conceive of a life different'' is the central element by which the method operates. By studying the motives behind the attack on Frazer we can make sense of Wittgenstein's use of fabricated anthropological scenarios. 
 
Crucial to this study are three claims Wittgenstein makes one paragraph below the passage just quoted. First, that anything we could come up with when imagining possible practices might just exist.
\begin{quote}
One sees how misleadig Frazer's explanations are [...] by noting that one could very easily invent primitive practices oneself, and it owuld be pure luck if they were not actually found somewhere.
\end{quote}
So, Wittgenstein notes, there is no principled difference between his thought experiments about how people could behave, a society could be constituted etc. and actual anthropological data. Imagining a practice serves just as well as actually observing it.
The second claim holds that there is a \textit{principle}, likely rooted in the underlying traits of human beings performing the rituals, to which the rituals accord. There is, thus, some pattern or \textit{general principle} which all ritualistic practices abide by; it threads through all societies and their ritual acts. This quote immediately follows the one about equivalence of possible and actual practices:
\begin{quote}
That is, the principle according to which these practices are arranged is a much more general one than in Frazer's explanation and it is present in our own minds, so that we ourselves could think up all the possibilities.
\end{quote}
The third claim is related to the first and can be found in the latter part of the quote just given. The imagined scenarios we may construct somehow instantiate or follow the same \textit{general principles} embedded in all ''our own minds''.

When putting these claims together we obtain a valid tool to investigate ''the principle according to which these practices are arranged'' of all the ritualistic practices of humankind. Without even having to check whether some conjured up example is or was actually existing at some point and place we can investigate the overarching, structural commonalities of all these rituals. The reason this is a valid and generally applicable method, according to Wittgenstein, is because we - human beings - are all in possession of the same kind of mind and, in virtue of this equally powerful mind, in the position to imagine all possible practices.
%\textbf{Explains palpable irateness Wittgenstein has towards Frazer. Because they are essentially in the same business - explaining humans - and they're data is, in Wittgenstein's view, equally valid, their differences in analysis are directly comparable.}
\footnote{Wittgenstein's exploration of this ramification of his thought is only very cursory, especially considering how bold it is. We might conjecture that some remnant from the Tractatus' idea that thought and world are structurally isomorphic lies behind Wittgenstein's confidence that all people equally possess this universal extension of imaginative powers.}

This reading presupposes a reading of ''our'' that allows for no principled difference between different kinds of human minds. One could debate whether Wittgensteins juxtaposition of ''us'' and ''them'', that is, ''the savages'', in other parts of his work (cf. PI 194, the next quote below) should play a role here. I, however, have the feeling that particularly on this occasion, amidst an attack on Frazer for tracing differences where no genuine ones are to be found, the ''our'' should be read in an all-encompassing, uniting sense. There are various passages dispersed throughout the remarks on Frazer where he argues for the fundamental likeness between ''us'' and ''the savages''. In some passage he speaks of ''kinship'' between people and he seems to mock Frazer at time for trying to construct a difference where there really is none to be found. 

Shortly after the threefold claim concerning the universal access to the realm of possible rituals he offers the following commonality between all humans. It is especially pertinent to the argument because of its concreteness as to what commonalities there could be and how these could lead to different people inventing practices that are similar in fundamental structure or \textit{general principle}.
\begin{quote}
There are dangers connected with eating and drinking, not only for savages, but also for us; nothing is more natural than the desire to protect oneself from these; and now we could devise such a preventative measure ourselves. - But according to what principle are we to invent them? Obviously, according to the one by which all dangers are reduced to the form of a few very simple ones which are immediately evident to man.
\end{quote}
Here the universally human need to engage with food and drink creates some ''immediately evident'' and ''simple'' principles known to the whole of mankind. These univeral principles then serve as the strucure according to which all actual practices as well as all imagined practices concerning nourishment must be built.

\section{Conclusion}
\hypertarget{sec4}{arrg}
%\textbf{That is, one could begin a book on anthropology by saying: When one examines the life and behavior of mankind throughout the world, on sees that, except for what might be called animal activities, such an ingestion, etc., etc., etc., men also perform actions which bear a characteristic peculiar to themselves, and these could be called ritualistic actions.}

\section{References}
\hypertarget{sec5}{grr}
%\printbibliography
\end{document}