\documentclass{article}
\title{Wittgenstein as an Antropologist\\
\large Subtitle}
\date{}
\author{Silvan Hungerbuehler}
\usepackage{enumitem}
\usepackage{verbatim}
\usepackage{hyperref}
\usepackage{comment}
\usepackage[utf8]{inputenc}
\usepackage[english]{babel}
\usepackage{amsmath}


%\usepackage[backend=biber]{biblatex}
%\addbibresource{references_wiggy.bib}
%\usepackage{csquotes}
\begin{document}
\maketitle
\section{Introduction}
Throughout his much of his later writing Wittgenstein frequently assumes what might be called an \textit{ethnological position}. In these moments Wittgenstein creates the illusion of the adventurous explorer visiting an entirely foreign people. In this hypothesiszed position of a complete stranger
%5Here I could use a good amount of Quine/Davidson/Glock description 
he then investigates some completely unknown language, as in PI 206: “[…] Denke, du kämst als Forscher in ein unbekanntes Land mit einer dir gänzlich fremden Sprache.” or in OC PI 207: “Denken wir uns, die Leute in jenem Land verrichteten gewöhnliche menschliche Tätigkeiten und bedienen sich dabei, wie es scheint, einer artikulierten Sprache.”. At various times he uses the picture of the wild tribe, as in OC 264: “Ich könnte mir den Fall denken, dass Moore von einem wilden Volksstamm gefangen genommen wird[…].” 

At the same time, Wittgenstein wrote an extensive critique of James George Frazer's monumental comparative study of religion and mythology. Frazer's account is definitely an anthroplogical one and some Wittgenstein's remarks thus of a more practical nature. The purpose of the essay at hand is to asses and compare Wittgenstein's views on the actual practice of anthropology - as manifested in his \textit{Remarks on Frazer's Golden Bough} - with the his own use of speculative ethnology. Wittgenstein has made a couple of interesting points delimiting what ethnological work can and can’t do when investigating ritual, mythology and belief. My question then is: How does Wittgenstein’s use of these ethnological thought experiments relate to his remarks on Frazer’s actual ethnographic work?

Since Wittgenstein is certainly no professional anthropologist, a first pertinent question must concern the purpose of his hypothesisez ethnological field trips. In \hyperlink{sec2}{Section 2} I will try to give some rough account of the use Wittgenstein makes of this philosophical device. I will also cursorely come to speak of his reception in the field of anthroplogy. \\
\hyperlink{sec3}{Section 3} will be concerned with tracing Wittgenstein's \textit{Remarks on Frazer's Golden Bough} and their relation to the findings of the previous section. Can Wittgenstein’s views on the proper place of anthropological description as well as his own (albeit merely supposed) ethnographic investigations be accommodated within his wider philosophical views? Finally, \hyperlink{sec4}{Section 4} concludes and \hyperlink{sec4}{Section 5} contains the bibliography.\\
\textbf{So far $\approx$ 370 words} 
\section{Philosophy as Anthropogy}
\paragraph{} Wittgenstein's later philosophy offers a powerful conception of human language use as a variety of rule-governed activities in human communities. Within and for these \textit{language-games}, interlocutors learn the correct application of words and response to their uttering, how to give explanations appropriate to the context etc. In short, the learning of a language thus consists in mastering the technique its usage. \\
''Am Anfang war die Tat.''\\
Just as human beings use tools in the context of their everyday activities, language users perform acts with wordsin the context of their communal lives.\\
''Das Wort "Sprachspiel" soll hervorheben, dass das \textit{Sprechen} der Sprache ein Teil ist einer T{\"a}tigkeit, oder einer Lebensform.''\\ %\cite{wittgenstein2009philosophische} \\
So human language is conceived to be centered in practice. Quite naturally, this practice takes place in the \textit{flow of life}, in the human community of family at first and broader society later. The concepts employed across different places, times, communities or societal subgroups arise from a shared way of living. From elementary human needs to highly contingent circumstances of one particular form of co-existence and evertything between these extremes, all these factors go into producing the particular concepts used by some group of humans.

The reason Wittgenstein frequently recurs to an \textit{anthropological} or \textit{ethnological} stance when describing and reflecting on philosophical issues is intimitaly tied to this conception of human language. For the fact that words and phrases have meaning at all, is tied to the fact that there are rules governing the correct us of words. If there were no standards to discern correct from incorrect application of a word, it would be meaningless. %(Cf.\cite{glock1996safari})
Thus acquiring and using a language is in a very important sense a normative practice. The community of language users constantly shapes what the language-games in use, expands old ones and abolishes moves deemed once legal.

%The logical step, then, is to study language as an anthropologist - as somebody who studies humans.

\paragraph{} So \textbf{concepts} are human creations, governed by rules stipulated, controled and inculcated by humans. As P.M. Hacker points out, however, it seems that Wittgenstein is not primarily interested in analysing the actual history of the concepts familiar to him (a possible exception might be his work in \textit{On Certainty}). So although he argues for the tantamount importance of viewing the grammar of the language we happen to do philosophy with as a historical product, he does not seem to see himself as a historian of that grammar. Hacker uses the slogan: ''Historicism without history."\\
Wittgenstein reflects on the relation between concept-formation and historical considerations in the Philosophy of Psychology:
%\cite{hacker2010wittgenstein}


\hypertarget{sec2}{Hacker:}
\begin{enumerate}
\item Concepts\\
Hacker: That human beings use language, engage in language-games, perform acts of speech in the context of their activities - these are anthropological facts about the nature history of man. What warrants using the epithets ''ethnological approach'' or ''anthropological approach'' in describing Wittgenstein's later philosophy is the perspective from which he view conceptual matters. 
\end{enumerate}
\section{Anthropological Remarks on Frazer}
\hypertarget{sec3}{uggh}
\section{Conclusion}
\hypertarget{sec4}{arrg}
\section{References}
\hypertarget{sec5}{grr}
%\printbibliography


\end{document}