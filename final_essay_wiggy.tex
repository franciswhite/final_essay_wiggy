\documentclass{article}
\title{Wittgenstein as an Antropologist\\
\large Subtitle}
\date{}
\author{Silvan Hungerbuehler}
\usepackage{enumitem}
\usepackage{verbatim}
\usepackage{hyperref}
\usepackage{comment}
\usepackage[utf8]{inputenc}
\usepackage[english]{babel}


%\usepackage[backend=biber]{biblatex}
%\addbibresource{references_wiggy.bib}

\begin{document}
\maketitle
\section{Introduction}
Throughout his much of his later writing Wittgenstein frequently assumes what might be called an \textit{ethnological position}. In these moments Wittgenstein creates the illusion of the adventurous explorer visiting an entirely foreign people. In this hypothesiszed position of a complete stranger
%5Here I could use a good amount of Quine/Davidson/Glock description 
he then investigates some completely unknown language, as in PI 206: “[…] Denke, du kämst als Forscher in ein unbekanntes Land mit einer dir gänzlich fremden Sprache.” or in OC PI 207: “Denken wir uns, die Leute in jenem Land verrichteten gewöhnliche menschliche Tätigkeiten und bedienen sich dabei, wie es scheint, einer artikulierten Sprache.”. At various times he uses the picture of the wild tribe, as in OC 264: “Ich könnte mir den Fall denken, dass Moore von einem wilden Volksstamm gefangen genommen wird[…].” 

At the same time, Wittgenstein wrote an extensive critique of James George Frazer's monumental comparative study of religion and mythology. Frazer's account is definitely an anthroplogical one and some Wittgenstein's remarks thus of a more practical nature. The purpose of the essay at hand is to asses and compare Wittgenstein's views on the actual practice of anthropology - as manifested in his \textit{Remarks's on Frazer's Golden Bough} - with the his own use of speculative ethnology. 

Since Wittgenstein is certainly no professional anthropologist, a first pertinent question must concern the purpose of his hypothesisez ethnological field trips? In \hyperlink{sec2}{Section 2} I will try to give some rough account of the use Wittgenstein makes of this philosophical device. I will also brieflyQuite naturally a second question emerged: How does Wittgenstein’s use of these ethnological thought experiments relate to his remarks on Frazer’s actual ethnographic work? As far as I recall Wittgenstein has made a couple of interesting points delimiting what ethnological work can and can’t do when investigating a ritual.  This led me to third point: Can Wittgenstein’s views on the proper place of anthropological description as well as his own (albeit merely supposed) ethnographic investigations be accommodated within his wider philosophical views?
Due to a a shortage of both time and space I would certainly not be able to give a detailed exegesis of Wittgenstein supposed ventures to the “wild tribe”. I would thus stick a more anecdotal treatment of these passages.
I already surveyed some of the literature in anthropology that attempts to work with Wittgenstein’s views (Veena Das: Wittgenstein and Anthropolgy; Dale Jacquette: Wittgenstein’s anthropologism in logic, philosphy and the social sciences;Philippe de Lara: Wittgenstein as Anthropologist: The Concept of Ritual Instinct) as well as Peter Hacker’s article: Wittgenstein’s Antropological and Ethnological Approach. Some of it is interesting and helpful, especially the anthropologist’s methodological considerations, yet I doubt that I will be able to get a decent overview quickly.
\section{Philosophy as Anthropogy} 
\hypertarget{sec2}{}
\section{Anthropological Remarks on Frazer}
\section{Conclusion}
\section{References}
%\printbibliography


\end{document}